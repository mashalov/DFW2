\documentclass[]{article}

\usepackage{geometry}
\geometry{
	a4paper,
	total={170mm,257mm},
	left=20mm,
	top=20mm,
}

\usepackage[utf8]{inputenc}
\usepackage[english,russian]{babel}
\usepackage{pscyr}
\usepackage{gensymb}
\usepackage{amsmath}
\usepackage{multicol}
\usepackage{epstopdf}
\usepackage{graphicx}
\begin{document}
%\begin{multicols}{2}	
	
Уравнения генератора
\begin{subequations}
\begin {equation}
M_j = T_j P_{nom}
\end {equation}
\begin {equation}
M_j\frac{ds}{dt} = (\frac{P_t}{1+s} - \frac{P}{1+s_v} - sK_{demp})
\end {equation}
\begin {equation}
\frac{d\delta}{dt} = \omega_0s
\end {equation}
\end{subequations}

\begin{equation}
E_{qenom}=\frac{U^4_{nom}+U^2_{nom}Q_{nom}(x_d+x_q)+S^2_{nom}x_dx_q}{U_{nom}\sqrt{U^4_{nom}+2U^2_{nom}Q_{nom}x_q+S^2_{nom}x^2_q}}
\end{equation}


До  двадцати лет я всерьез интересовалась музыкой. Я прилежно училась и
прилежно играла но вскоре это показалось бессмысленным, мать уже умерла и не
было невыносимой печали, но не было  настоящего интереса который побуждал бы
меня продолжать. В  повести  Ада из Географии и  пьесы Гертруда Стайн  очень
точно описала меня какой я была в те годы.
Следующие лет шесть я была вполне занята. Я  вела приятный образ жизни,
у меня было много друзей много развлечений много интересов, моя жизнь была в
меру полной и приносила мне удовольствие $ \alpha=sin(\beta+\frac{\pi}{2}) $ но я была в ней не особенно горяча.
Так  я  подхожу к пожару  в  Сан-Франциско  из-за  которого  в Сан-Франциско
приехал  из Парижа старший  брат Гертруды Стайн со  своей женой и их  приезд
совершенно перевернул мою жизнь.
Тогда я жила с отцом  и братом. Мой отец был спокойный человек  который
все принимал спокойно, хотя переживал  глубоко. В первое ужасное утро пожара
в Сан-Франциско я  разбудила его и  сказала, было землетрясение  и  сейчас в
городе  пожар.  Заработаем   себе  дурную  славу  на  Востоке,   ответил  он
поворачиваясь на другой бок и засыпая опять. Помню однажды, когда мой брат с
товарищем
- 8 -
поехали кататься верхом и лошадь  кого-то из них вернулась к  гостинице
без седока, мать  товарища  начала  закатывать жуткую  истерику. Успокойтесь
сударыня, сказал отец, может быть это мой сын разбился. Одно его непреложное
правило  я  помню всегда, если вынужден что-то делать, делай охотно.  Еще он
меня учил что хозяйке никогда не нужно  извиняться  за то что в доме  у  нее
где-то беспорядок, ни малейшего беспорядка, коль скоро хозяйка есть, нет.
Как я уже говорила,  нам  очень удобно  жилось всем вместе,  и  у  меня
никогда  не бывало ни  мыслей  о переменах, ни  сильного  желания перемен. С
нарушением привычного  течения нашей жизни пожаром а затем приездом старшего
брата Гертруды Стайн и его жены стало иначе.
Мисс Стайн привезла с собой  три  небольшие картины  Матисса, те первые
современные живописные вещи, которые пересекли  Атлантику. Я познакомила
%\end{multicols}
\end{document}
