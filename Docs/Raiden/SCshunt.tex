\documentclass[lettersize,journal]{IEEEtran}
\usepackage{amsmath,amsfonts}
\usepackage{algorithmic}
\usepackage{algorithm}
\usepackage{array}
\usepackage[caption=false,font=normalsize,labelfont=sf,textfont=sf]{subfig}
\usepackage{textcomp}
\usepackage{stfloats}
\usepackage{url}
\usepackage{verbatim}
\usepackage{graphicx}
\usepackage{cite}
\usepackage{tabularx}
\usepackage{boldline}
\usepackage{empheq}
\usepackage{amssymb}
%\usepackage[justification=centering]{caption}
%\usepackage{hyperref}
\hyphenation{op-tical net-works semi-conduc-tor IEEE-Xplore}
% updated with editorial comments 8/9/2021
%\usepackage[russian]{babel}
%\usepackage[utf8]{inputenc}
%\usepackage{pscyr}
\usepackage[T2A]{fontenc}

\DeclareMathOperator*{\argmin}{arg\,min} % thin space, limits underneath in displays
\newcommand\norm[1]{\left\lVert#1\right\rVert}

\makeatletter
\def\endthebibliography{%
	\def\@noitemerr{\@latex@warning{Empty `thebibliography' environment}}%
	\endlist
}
\makeatother

\begin{document}
	
\onecolumn

%\title{Raiden. Power System Transient Stability Simulation Software Prototype}

%\author {Eugene Mashalov, Joint Stock Company "Scientific and Technical Center of Unified Power System", Ekaterinburg, Russia}


%\markboth{Journal of \LaTeX\ Class Files,~Vol.~14, No.~8, August~2021}%
%${Shell \MakeLowercase{\textit{et al.}}: A Sample Article Using IEEEtran.cls for IEEE Journals}

%\IEEEpubid{0000--0000/00\$00.00~\copyright~2021 IEEE}
% Remember, if you use this you must call \IEEEpubidadjcol in the second
% column for its text to clear the IEEEpubid mark.

%\maketitle

%\begin{abstract}
%This paper discusses implementation details of the transient stability analysis prototype %software "Raiden", based on the implicit integration scheme. The main advantages of an %implicit integration scheme with combined treatment of differential and algebraic variables %as well as an integration process with discrete event handling are considered. The results %of test simulations and their comparison with existing software are presented.
%\end{abstract}

%\begin{IEEEkeywords}
%Transient Stability Analysis, Implicit Integration, Discontinuous DAE
%\end{IEEEkeywords}

\section{Расчет шунта по остаточному напряжению и R/X}
Добавим в уравнение баланса тока узла \(i\) проводимость шунта \(y_s\):
\begin{equation}
	\label{eqn_current_balance}
	(y_{ii} + y_s)v_i + \sum_{j}{v_jy_{ij}} = I_i
\end{equation}
и рассчитаем его значение из условия:
\begin{equation}
 \label{eqn_Us_magnitude}
 |v_i|=U_s
\end{equation} 
полагая, что исходно \(y_s\) задан в форме сопротивления с отношением \(k=R_s/X_s\):
\begin{equation}
	\label{eqn_conductance}
	\begin{split}
	y_s = \frac{1}{R_s+jX_s}=&\frac{1}{X_s(k+j)}=\frac{k}{X_s(k^2+1)} - \frac{j}{X_s(k^2+1)}=b_s(k-j) \\ b_s =& \frac{1}{X_s(k^2+1)}
	\end{split}
\end{equation}
Примем на итерации
 \(\sum_{j}{v_jy_{ij}}=const\) и \(I_i=const\) и обозначим:
\begin{equation}
	\label{eqn_composite_current}
	I'= I_i - \sum_{j}{v_jy_{ij}}
\end{equation}

\begin{equation}
	\label{eqn_implicit_ys}
	(y_{ii} + y_s)v_i = I'
\end{equation}
Перейдем к составляющим:

\begin{equation}
	\label{eqn_Vre}
	v_{iRe} = \frac{I'_{Re}(g_{ii}+kb_s) + I'_{Im}(b_{ii}-b_s)}{(b_{ii}-b_s)^2+(g_{ii}+kb_s)^2}
\end{equation}
\begin{equation}
	\label{eqn_Vim}
	v_{iIm} = \frac{I'_{Im}(g_{ii}+kb_s) - I'_{Re}(b_{ii}-b_s)}{(b_{ii}-b_s)^2+(g_{ii}+kb_s)^2}
\end{equation}

Используя условие (\ref{eqn_Us_magnitude}) 
\begin{equation}
	\label{eqn_Vre_Vim_Abs}
	v_{iRe}^2+v_{iIm}^2 = U_s^2
\end{equation}
и с учетом (\ref{eqn_Vre}), (\ref{eqn_Vim}) получим:
\begin{equation}
	\label{eqn_bs1}
	\frac{I_{Re}^{'2}+I_{Im}^{'2}}{(b_{ii}-b_s)^2+(g_{ii}+kb_s)^2} = U_s^2
\end{equation}
\(b_s\) является корнем уравнения:
\begin{equation}
	\label{eqn_bs2}
	-b_s^2(k^2+1) + b_s2(b_{ii} - kg_{ii}) + \left(\frac{I_{Re}^{'2}+I_{Im}^{'2}}{U_s^2} - g_{ii}^2 - b_{ii}^2\right)=0
\end{equation}

Для выбора корней \(b_{s1}\) и \(b_{s2}\) можно воспользоваться условием:
\begin{equation}
	\label{eqn_vx}
	v_i(x) = \frac{I}{g_{ii} + kx + i(b_{ii} - x)}
\end{equation}

\begin{equation}
	\label{eqn_bs1_bs2_select}
	b_s=\argmin_{x\in{b_{s1}, b_{s2}}}{(\norm{v_i - v(x)}_2)}
\end{equation}
то есть выбирать на итерации шунт \(b_s\), который дает вектор \(v_i(b_s)\) наиболее близкий к текущему \(v_i\).
На итерации целесообразно зафиксировать \(v_i\)=const, для чего (\ref{eqn_current_balance}) можно представить в виде:
\begin{equation}
	\label{eqn_vi_fix}
	v_i=v_i(b_s)
\end{equation}
При этом само уравнение технически проще оставить в СЛУ, так напряжение \(v_i\) будет итерационно уточняться. 
\end{document}


