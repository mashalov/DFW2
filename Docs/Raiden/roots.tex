\documentclass[lettersize,journal]{IEEEtran}
\usepackage{amsmath,amsfonts}
\usepackage{algorithmic}
\usepackage{algorithm}
\usepackage{array}
\usepackage[caption=false,font=normalsize,labelfont=sf,textfont=sf]{subfig}
\usepackage{textcomp}
\usepackage{stfloats}
\usepackage{url}
\usepackage{verbatim}
\usepackage{graphicx}
\usepackage{cite}
\usepackage{tabularx}
\usepackage{boldline}
\usepackage{empheq}
\usepackage{amssymb}
%\usepackage[justification=centering]{caption}
%\usepackage{hyperref}
\hyphenation{op-tical net-works semi-conduc-tor IEEE-Xplore}
% updated with editorial comments 8/9/2021
%\usepackage[russian]{babel}
%\usepackage[utf8]{inputenc}
%\usepackage{pscyr}
\usepackage[T2A]{fontenc}
\usepackage{steinmetz}

\DeclareMathOperator*{\argmin}{arg\,min} % thin space, limits underneath in displays
\newcommand\norm[1]{\left\lVert#1\right\rVert}

\makeatletter
\def\endthebibliography{%
	\def\@noitemerr{\@latex@warning{Empty `thebibliography' environment}}%
	\endlist
}
\makeatother

\begin{document}
	
\onecolumn

%\title{Raiden. Power System Transient Stability Simulation Software Prototype}

%\author {Eugene Mashalov, Joint Stock Company "Scientific and Technical Center of Unified Power System", Ekaterinburg, Russia}


%\markboth{Journal of \LaTeX\ Class Files,~Vol.~14, No.~8, August~2021}%
%${Shell \MakeLowercase{\textit{et al.}}: A Sample Article Using IEEEtran.cls for IEEE Journals}

%\IEEEpubid{0000--0000/00\$00.00~\copyright~2021 IEEE}
% Remember, if you use this you must call \IEEEpubidadjcol in the second
% column for its text to clear the IEEEpubid mark.

%\maketitle

%\begin{abstract}
%This paper discusses implementation details of the transient stability analysis prototype %software "Raiden", based on the implicit integration scheme. The main advantages of an %implicit integration scheme with combined treatment of differential and algebraic variables %as well as an integration process with discrete event handling are considered. The results %of test simulations and their comparison with existing software are presented.
%\end{abstract}

%\begin{IEEEkeywords}
%Transient Stability Analysis, Implicit Integration, Discontinuous DAE
%\end{IEEEkeywords}

\section{Решение уравнения (7)}
\begin{equation}
	t=\frac{u_{0k}/u_{kk} -1}{z_{kk}}
\end{equation}
Сопоставим равенство комплексных чисел по модулю и углу:
\begin{equation}
 	tz_{kk} + 1 = \frac{u_{0k}}{u_{kk}}
 \end{equation} 
учитывая \(Im(t) = 0\) для модулей:
\begin{equation}
 	(tz_{kkRe} + 1)^2 + tz_{kkIm}^2= \frac{|u_{0k}|^2}{|u_{kk}|^2}
\end{equation} 
\begin{equation}
	\label{eqn_quad}
	t^2|z_{kk}|^2 + 2tz_{kk} + 1 -  \frac{|u_{0k}|^2}{|u_{kk}|^2} = 0
\end{equation} 
и для углов:
\begin{equation}
		\arctan \left(\frac{tz_{kkIm}}{tz_{kkRe} + 1}\right) = \arg(u_{0k}) - \arg(u_{kk})
\end{equation} 
\begin{equation}
	\delta_{kk}(t) = \delta_{0k} - \arctan \left(\frac{tz_{kkIm}}{tz_{kkRe} + 1}\right)
\end{equation} 
\begin{equation}
	u_{kk}(t)=|u_{kk}|\phase{\delta_{kk}(t)}
\end{equation} 
Для выбора корней \(t_1\) и \(t_2\) можно воспользоваться условием:
\begin{equation}
	t=\argmin_{x\in{t_1, t_2}}{\norm{u_{kk} - u_{kk}(x)}_2}
\end{equation}
то есть выбирать на итерации шунт,  который дает вектор \(u_{kk}(t)\) наиболее близкий к текущему \(u_{kk}\).
\end{document}


